\documentclass{article}

\usepackage[french]{babel}
\usepackage[T1]{fontenc}
\usepackage[utf8]{inputenc}
\usepackage{mathtools}
\usepackage{amssymb}
\usepackage{amsthm}
\usepackage{upquote}
\usepackage{xspace}
\usepackage{xcolor}

\newcommand{\myname}{AUTHOR Dummy}
\renewcommand {\myname} {\name{Dummy}{Author}\xspace}

\newcommand{\mystyle}[2][]{{#1 #2}}

\newcommand{\name}[2]{\mystyle[\color{gray}\bfseries]{#1} \mystyle[\color{blue}\scshape]{#2}}

\title{\emph{Généalogie}}
\author{Dummy}
\date{\today}

\begin{document}
\maketitle

Je m’appelle \myname. \\
— \myname est le fils de \name{Hello}{Author}.\\
— \myname est le petit-fils de \name{Friend}{Author}.\\
— \name{Ros}{Chill} est le neveu de \myname.\\
— \name{Funny}{Chill} est le petit frère de \name{Ros}{Chill}  et donc également le neveu de \myname.\\

Ça,\mystyle[\bfseries\slshape]{c’est} du\mystyle[\scshape\large]{style}!


\end{document}
